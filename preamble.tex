
\usepackage{amsmath}
\usepackage{amsfonts}
\usepackage{amssymb}    		%为了更好的符号,如
\usepackage[version=4]{mhchem}	 %简单化学式
\usepackage{chemfig}	   	%复杂化学式
\usepackage{graphicx}
\usepackage{pifont}			%用于
\usepackage{makeidx}        %调用makedidx宏包
\usepackage{listings}       %用于插入代码块
\usepackage{xcolor}         %为代码块设置颜色

\lstset{
	columns=fixed,     	                                % 在左侧显示行号
	frame=single,                                       % 不显示背景边框
	backgroundcolor=\color[RGB]{245,245,244},           % 设定背景颜色
	keywordstyle=\color[RGB]{40,40,255},                % 设定关键字颜色
	numberstyle=\footnotesize\color{darkgray},          % 设定行号格式
	commentstyle=\it\color[RGB]{0,96,96},               % 设置代码注释的格式
	stringstyle=\rmfamily\slshape\color[RGB]{128,0,0}, 	% 设置字符串格式
	showstringspaces=false,                             % 不显示字符串中的空格
	language=TeX,                                       % 设置语言
	escapeinside=``										% 逃逸字符
}														%
\setcounter{secnumdepth}{4}								%设置标题的等级深度为4
\usepackage[bookmarkstype=toc,
	bookmarksopen=true,
	breaklinks,
	colorlinks,
	linkcolor=black,
	citecolor=black,
	urlcolor=black,
	bookmarksnumbered=true,
	bookmarksopenlevel=4]{hyperref}						%生成标签,就是大纲也是目录	
\ctexset{chapter ={
		beforeskip = 0pt, 
		fixskip = true, 
		format = \Huge\bfseries, 
		nameformat = \rule{\linewidth}{2bp}\par\bigskip\hfill\chapternamebox, 
		number = \arabic{chapter}, 
		aftername = \par\medskip, 
		aftertitle = \par\bigskip\nointerlineskip\rule{\linewidth}{2bp}\par}
		}												%章节格式设置
\usepackage[a4paper,left=3cm,right=3cm]{geometry}
\newcommand\chapternamebox[1]{% 
	\parbox{\ccwd}{\linespread{1}\selectfont\centering #1}} 	

\newcommand\superequiv{\mathrel{\rlap{\raisebox{\fontdimen22\textfont2}{$=$}}\raisebox{-0.5\fontdimen22\textfont2}{$ = $}}} %4横线	