\chapter{土壤农化分析的基础知识}

学习土壤农化分析,和学习其他课程一样, 必须掌握有关的基本理论、基本知
识和基本操作技术。基本知识包括与土壤农化分析有关的数理化知识、分析实验
室知识、农业生产知识和土化专业知识。这些基本知识必须在有关课程的学习中
以及在生产实践和科学研究工作中不断吸取和积累。本章只对土化分析用的纯水
、试剂、器皿等基本知识作一简要的说明。定量分析教材中的内容一般不再重复。

\section{土壤农化分析用纯水}
\subsection{纯水的制备}

分析工作中需用的纯水用量很大,必须注意节约、水质检查和正确保存,勿使受
器皿和空气等来源的污染,必要时装苏打-石灰管防止\ce{CO2}的溶解沾污。纯
水的制备常用蒸馏法和离子交换法。蒸馏法是利用水与杂质的沸点不同,经过外
加热使所产生的水蒸气经冷凝后制得。蒸馏法制得的蒸馏水,由于经过高温的处
理,不易长霉;但蒸馏器多为铜制或锡制,因此蒸馏水中难免有痕量的这些食金
属离子存在。实验室自制时可用电热蒸馏水器,出水量有5、10、20、或50L/h等
种,使用尚称方便,但耗电较多,出水速度较小。工厂和浴室利用废蒸汽所得的
副产蒸馏水,质量较差,必须检查后才能使用。

离子交换法可制得质量较高的纯水——去离子水,一般是用自来水通过离子纯水器
制得,因未经高温灭菌,往往容易长霉。离子交换纯水器可以自己装置,各省市
也有商品纯水器供应。

水通过交换树脂获得的纯水称离子交换水或去离子水。离子交换树脂是一种不溶
性的高分子化合物。组成树脂的骨架部分具有网状结构,对酸碱及一般溶剂相当
稳定, 而骨架上又有能与溶液中阳离子或阴离子进行交换的活性基团。在树脂
庞大的结构中 磺酸基(\ce{SO3-} \ce{H+})或季铵基[\ce{CH2N+}  
\ce{(CH3)3OH-},简作$\superequiv$\ce{N+} \ce{OH-}]等是活性基团, 其
余的网状结构是树脂的骨架,可以用R表示。上述两种树脂的结构可简写为
\ce{R\superequiv SO3H}和\ce{R$\superequiv$ NOH}。当水流通过装有离子
交换树脂的交换器时,水中的杂质离子被离子交换树脂所截留。这是因为离子交
基中的\ce{H+}或\ce{OH-}与水中的杂质离子(如\ce{Na+} ,\ce{Ca2+},
\ce{Cl-},\ce{SO4-})交换,交换下来的\ce{H+}与\ce{OH-}结合为\ce{H2O}
而杂质离子则被吸附在树脂上,以阳离子\ce{Na+}和阴离子\ce{Cl-}为例,
其化学反应式为:

{\centering
\ce{R \superequiv SO_{3}H + Na+ <=>   R-SO3Na + H+}\\
\ce{R \superequiv NOH + Cl- <=> R \superequiv NCl + OH-}\\
\ce{OH- + H- -> H2O}\\}

上述离子交换反应是可逆的,当\ce{H+}与\ce{OH-}的浓度增加到一定程度时,
反应向相反方向进行,这就是离子交换树脂再生的原理。在纯水制造中,通常采
用强酸性阳离子交换树脂(如国产732树脂)和强碱性阴离子树脂(如国产717树脂)
。新的商品树脂一般是中性盐型式的树脂(常制\ce{R-SO3Na}和\ce{R \superequiv NCl}
等型式),性质较稳定、便于贮存。在使用之前必须进行净化和转型处理,使之
转化为所需的\ce{H+}型和\ce{OH-}型的树脂。

离子交换树脂的性能与活性基团和网状骨架、树脂的粒度和温度、pH等有关。
\ding{172} 活性基团越多,交换容量越大。一般树脂的交换容量为3$\sim$6 
mol·kg$^{-1}$,干树脂(离子型式)。活性基团和种类不同,能交换的离子基
团也不同。\ding{173}网状骨架的网眼是由交联剂形成的。例如上述苯乙烯系离
子交换树脂结构中的长碳链,是由若干个苯乙烯聚合而成。长链之间则用二乙希
苯交联起来,二乙烯苯就是交联剂。树脂骨架中所含交联剂的质量百分率就是交
联度。交联度小时,树脂的水溶性强,泡水后的膨胀性大, 网状结构的网眼大、
交换速度快,大小离子都容易进人网眼,交换的选择性低。反之,交联度大时,
则水溶性弱,网眼小,交换慢,大的离子不易进人,具有一定的选择性。制备纯
水的树脂,要求能除去多种离子,所以交联度要适当小。但同时以要求树脂难溶于
水,以免沾污纯水,所以交联度又要适当地大。实际选用时,交联度以7\%~12\%
为宜。
\ding{174}树脂的粒度越小(颗粒越小),工作交换量(实际上能交换离子的最大量
)越大,但在交换柱中充填越紧密,流速就越慢。制备纯水用的树脂粒度度以在0.3
$\sim$1.2mm(50$\sim$16目)之间为宜。
\ding{175}温度过高或过低,对树脂的强度和交换容量都有很大的影响。温度降低
时,树脂的交换容量和机械强度都随之降低;冷至