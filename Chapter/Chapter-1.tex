\chapter{土壤农化分析的基础知识}

学习土壤农化分析,和学习其他课程一样, 必须掌握有关的基本理论、基本知识和基本操作技术。基本知识包括与土壤农化分析有关的数理化知识、分析实验室知识、农业生产知识和土化专业知识。这些基本知识必须在有关课程的学习中以及在生产实践和科学研究工作中不断吸取和积累。本章只对土化分析用的纯水、试剂、器皿等基本知识作一简要的说明。定量分析教材中的内容一般不再重复。

\section{土壤农化分析用纯水}
\subsection{纯水的制备}

分析工作中需用的纯水用量很大,必须注意节约、水质检查和正确保存,勿使受器皿和空气等来源的污染,必要时装苏打-石灰管防止\ce{CO2}的溶解沾污。纯水的制备常用蒸馏法和离子交换法。蒸馏法是利用水与杂质的沸点不同,经过外加热使所产生的水蒸气经冷凝后制得。蒸馏法制得的蒸馏水,由于经过高温的处理,不易长霉;但蒸馏器多为铜制或锡制,因此蒸馏水中难免有痕量的这些食金属离子存在。实验室自制时可用电热蒸馏水器,出水量有5、10、20、或50L/h等种,使用尚称方便,但耗电较多,出水速度较小。工厂和浴室利用废蒸汽所得的副产蒸馏水,质量较差,必须检查后才能使用。

离子交换法可制得质量较高的纯水——去离子水,一般是用自来水通过离子纯水器制得,因未经高温灭菌,往
往容易长霉。离子交换纯水器可以自己装置,各省市也有商品纯水器供应。

水通过交换树脂获得的纯水称离子交换水或去离子水。离子交换树脂是一种不溶性的高分子化合物。组成树脂的骨架部分具有网状结构,对酸碱及一般溶剂相当稳定, 而骨架上又有能与溶液中阳离子或阴离子进行交换的活性基团。在树脂庞大的结构中 磺酸基(\ce{SO3-} \ce{H+})或季铵基[\ce{CH2N+}  \ce{(CH3)3OH-},简作$\superequiv$\ce{N+} \ce{OH-}]等是活性基团, 其余的网状结构是树脂的骨架,可以用R表示。上述两种树脂的结构可简写为\ce{R\superequiv SO3H}和\ce{R$\superequiv$ NOH}。当水流通过装有离子交换树脂的交换器时,水中的杂质离子被离子交换树脂所截留。这是因为离子交换基中的\ce{H+}或\ce{OH-}与水中的杂质离子(如\ce{Na+} ,\ce{Ca2+},\ce{Cl-},\ce{SO4-})交换,交换下来的\ce{H+}与\ce{OH-}结合为\ce{H2O}而杂质离子则被吸附在树脂上,以阳离子\ce{Na+}和阴离子\ce{Cl-}为例,其化学反应式为:
{\centering
\ce{R \superequiv SO_{3}H + Na+ <=>   R \superequiv Na + H^+}\\
\ce{R \superequiv NOH + Cl-<=> R\superequiv NC + OH-}\\
\ce{OH- + H- -> H2O}\\}
上述离子交换反应是可逆的,当\ce{H+}与\ce{OH-}的浓度增加到一定程度时,反应向相反方向进行,这就是离子交换树脂再生的原理。在纯水制造中,通常采用强酸性阳离子交换树脂(如国产732树脂)和强碱性阴离子树脂(如国产717树脂)。新的商品树脂一般是中性盐型式的树脂(常制\ce{R-SO3Na}和\ce{R \superequiv NCl}等型式),性质较稳定、便于贮存。在使用之前必须进行净化和转型处理,使之转化为所需的\ce{H+}型和\ce{OH-}型的树脂。

离子交换树脂的性能与活性基团和网状骨架、树脂的粒度和温度、pH等有关。
\ding{172} 活性基团越多,交换容量越大。一般树脂的交换容量为3$\sim$6 mol/kg^{-1},干树脂(离子型式)。活性基团和种类不同,能交换的离子基团也不同。
\ding{173}网状骨架的网眼是由交联剂形成的。例如上述苯乙烯系离子交换树脂结构中的长碳链,是由若干个苯乙烯聚合而成。长链之间则用二乙希苯交联起来,二乙烯苯就是交联剂。树脂骨架中所含交联剂的质量百分率就是交联度。交联度小时,树脂的水溶性强,泡水后的膨胀性大, 网状结构的网眼大、交换速度快,大小离子都容易进人网眼,交换的选择性低。反之,交联度大时,则水溶性弱,网眼小,交换慢,大的离子不易进人,具有一定的选择性。制备纯水的树脂,要求能除去多种离子,所以交联度要适当小。但同时以要求树脂难溶于水,以免沾污纯水,所以交联度又要适当地大。实际选用时,交联度以7\%~12\%为宜。
\ding{174}树脂的粒度越小(颗粒越小),工作交换量(实际上能交换离子的最大量)越大,但在交换柱中充填越紧密,流速就越慢。制备纯水用的树脂粒度度以在0.3~1.2mm(50~16目)之间为宜。
\ding{175}温度过高或过低,对树脂的强度和交换容量都有很大的影响。温度降低时,树脂的交换容量和机械强度都随之降低;冷至